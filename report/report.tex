\documentclass[a4paper]{article}
\setlength{\parskip}{\baselineskip}
\let\oldbibliography\thebibliography
\renewcommand{\thebibliography}[1]{%
  \oldbibliography{#1}%
  \setlength{\itemsep}{0pt}%
}

% First load extension packages
\usepackage[a4paper,margin=25mm]{geometry}    % page layout
\usepackage{setspace} \onehalfspacing         % line spacing
\usepackage{amsfonts,amssymb,amsmath}         % useful math extensions
\usepackage{graphicx}                         % graphics import
\graphicspath{{../figs/}}
\pdfpageattr{/Group << /S /Transparency /I true /CS /DeviceRGB>>}
\usepackage[colorlinks=true, allcolors=black]{hyperref}
\usepackage{hhline}
\usepackage{subcaption}
% Change paragraph indentation
\setlength{\parskip}{10pt}
\setlength{\parindent}{0pt}
\usepackage[toc]{multitoc}
%\renewcommand*{\multicolumntoc}{2}
\setlength{\columnsep}{36pt}
% User-defined commands
\newcommand\numberthis{\addtocounter{equation}{1}\tag{\theequation}}
% topmatter
\title{\vspace{-50pt}\huge\bfseries Developing a Coarse-Grained Climate Model
to Investigate Long-Term Behaviour}

\author{Liam Wheen \\ Supervised by Dr O.\ Benjamin}

\date{\today}
\pagenumbering{gobble}
% main body
\begin{document}
\begin{titlepage}
\maketitle
\hrule
\vspace{-10pt}
\begin{abstract}
\noindent
This is the abstract.
\end{abstract}
\hrule
\end{titlepage}
%\vspace{20pt}
\newpage
\pagenumbering{gobble}
%\begin{spacing}{0.6}
\tableofcontents
%\end{spacing}
\newpage
\pagenumbering{arabic}
\section{Introduction}
\section{Budyko Model}
The Budyko model is an energy balance model (EBM) that explores the positive
feedback effect of polar ice cap albedo. The increased reflectivity of polar
ice caps causes Earth to receive less net energy from the sun across these
regions. If a sufficient amount of Earth's surface is covered in ice, 
global temperatures drop and the polar ice caps advance towards the equator.
Conversely, if the ice caps only exist at very high latitudes, then the reduced
reflectivity will cause Earth's temperature to rise, and this ice-line to
recede further.

A number of assumptions are made in order for this EBM to work. Temperature
is modelled as varying only across latitude, reducing the domain of the
system from a spherical surface, to a 1-dimensional temperature profile. In
order to simplify integral that span the latitudes, the
domain is described using $y = \sin(\varphi)$, where $\varphi$ is latitude.
To simplify the system further, the Earth is treated as an entirely water-based
planet. This allows for the differences in ground reflectivity to be ignored,
focusing only on the albedo of water, and ice. The Earth is also modelled as
symmetric across the equator, allowing for only one half of the domain to be
considered, i.e. $y\in [0,1]$. The impact of this last assumption is
investigated in section ----add ref----.

The dynamical equation for temperature consists of three components. They
describe the flow of energy into, out of, and around Earth.
The first of these is incoming solar radiation, insolation.
This depends on the irradiance given out by the sun, as well as the distance
and orientation of the Earth to the sun. The amount of insolation experienced
at different latitudes varies proportional to the
cosine of the angle between the ecliptic plane and the corresponding point on
Earth. This is why the poles remain permanently colder than the equator. 

The next component is the outgoing infra-red radiation.
This is approximated to linearly depend on the temperature of Earth. Although
the energy flow occurring within Earth's atmosphere is far more complex, 
empirical data was used to establish the net flow of heat from
Earth to the first order.

The final component of energy flow is the transport of energy around Earth, from
the hot equator to the relatively cooler poles. This depends on the difference
between the average temperature of the earth and the temperature at each
latitude.

Combining these elements gives the dynamical equation for temperature to be

\[
  R\frac{\partial T(y,t)}{\partial t} = \underbrace{Qs(y) (1-\alpha(y,\eta))}_{\text{Insolation}}
  - \underbrace{(A+BT(y,t))}_{\text{Reradiation}}
  + \underbrace{C(\bar{T}-T(y,t))}_{\text{Transport}},
\]

where $A, B, C,$ and $R$ are constants found using empirical data, $Q$ is the
annually averaged solar insolation experienced across the Earth, distributed
according to latitude with the function 

\[
  s(y) = 1+ \frac{1}{2}c_\beta(3y^2-1) 
\]

where $c_\beta = \frac{5}{16}(3\sin^2\beta - 2)$ and $\beta$ is Earth's axial
tilt or obliquity.


\begin{figure}
\centering
\begin{subfigure}{.5\textwidth}
  \centering
  \includegraphics[width=\linewidth]{demo_budyko_0.pdf}
  \caption{0 Years}
  \label{fig:sub1}
\end{subfigure}%
\begin{subfigure}{.5\textwidth}
  \centering
  \includegraphics[width=\linewidth]{demo_budyko_1.pdf}
  \caption{10 Years}
  \label{fig:sub2}
\end{subfigure}
\begin{subfigure}{.5\textwidth}
  \centering
  \includegraphics[width=\linewidth]{demo_budyko_2.pdf}
  \caption{3000 Years}
  \label{fig:sub3}
\end{subfigure}%
\begin{subfigure}{.5\textwidth}
  \centering
  \includegraphics[width=\linewidth]{demo_budyko_3.pdf}
  \caption{10000 Years}
  \label{fig:sub4}
\end{subfigure}
\caption{A figure with two subfigures}
\label{fig:test}
\end{figure}


\section{Insolation}
To implement an asymmetrical version of the Budyko model, the difference
between the northern and southern hemispheres must first be established.
Insolation is solar irradiance per unit time. It is the first component that must be
modelled to help identify how the two hemispheres differ. This simulation takes
the values for eccentricity, obliquity, and precession from the
forecasted data produced by Laskar \cite{milanko_data}.

With this data, the simulation can show how daily insolation varies across
latitudes and time. Figure \ref{fig:daily_ave_insol_all_lats} shows
insolation across all latitudes during the year 2000. The Earth's orbital
parameters at this point are shown in Table \ref{tab:orbital_params}.


\begin{table}[h]
  \centering
  \caption{Earth's orbital parameters on the year 2000.}
\label{tab:orbital_params}
\begin{tabular}{lllll}
\multicolumn{1}{l|}{Eccentricity:} & 0.0167 &  &  &  \\ \cline{1-2}
\multicolumn{1}{l|}{Obliquity:}    & 0.4091 radians &  &  &  \\ \cline{1-2}
\multicolumn{1}{l|}{Precession:}   & 2.9161 radians &  &  &  \\
                                   &        &  &  &
\end{tabular}
\end{table}

\begin{figure}
  \centering
  \includegraphics[width=\linewidth]{both_daily_ave_insolation_all_lats.pdf}
  \caption{Contours showing the daily average irradiance (W/m$^2$) arriving at
    the atmosphere for different latitudes over a year period. The left plot is
  based on current orbital parameters, as shown in Table
\ref{tab:orbital_params}, the right plot is how the irradiance will be
distributed 1.015$\times10^6$ years from now, when eccentricity is 0.0582}
  \label{fig:daily_ave_insol_all_lats}
\end{figure}

\begin{thebibliography}{30}
  \bibitem{milanko_data}
    Laskar J, Robutel P, Joutel F, Gastineau M, Correia AC, Levrard B. A
    long-term numerical solution for the insolation quantities of the Earth.
    Astronomy \& Astrophysics. 2004 Dec 1;428(1):261-85.
\end{thebibliography}
\end{document}
